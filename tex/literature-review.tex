\chapter{Literature Review}
The literature review\footnotemark is used to demonstrate how other people have addressed the problem that you have identified, and to show how you have used the existing body of work to develop your particular research project. If there is more than one topic to consider, a second chapter summarising secondary research may be required.

\section{Sections}
Different sections review the relevant knowledge that already exists on different aspects of the problem. If you are using a citation in the text, you might want the author name implicit as in this citation (Jones and Smith, 2005). Conversely, it might make the sentence flow better to include the authors explicitly, for example by mentioning the excellent results from Bloggs and Simpleton (2004) which corroborate your theory.
\subsection{Subsection}
Subsections can be used for more fine grained descriptions.
	
\section{Research question}
Based on the literature that you have reviewed, you should be able to hypothesise an answer to your research question, to be tested in the rest of the dissertation.

\footnotetext{Here is a footnote where you might mention something about the term literature review or whatever you need to make a footnote about. Note that we discourage footnotes as you can usually have this effect in the text with parentheses.  Use them sparingly.}

\section{Summary}
It can be helpful to summarise the literature review with an indication of how the main points in the literature have informed the direction that you will take in the rest of the dissertation.

